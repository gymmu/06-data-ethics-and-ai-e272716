\documentclass{report}

\usepackage[ngerman]{babel}
\usepackage[utf8]{inputenc}
\usepackage[T1]{fontenc}
\usepackage{hyperref}
\usepackage{csquotes}
\usepackage[a4paper]{geometry}

\usepackage[
    backend=biber,
    style=apa,
    sortlocale=de_DE,
    natbib=true,
    url=false,
    doi=false,
    sortcites=true,
    sorting=nyt,
    isbn=false,
    hyperref=true,
    backref=false,
    giveninits=false,
    eprint=false]{biblatex}
\addbibresource{../references/bibliography.bib}


\title{Ethik im Umgang mit Daten}
\author{Jelena Jovanoski}
\date{\today}


\begin{document}

\maketitle

\abstract{
    Dies ist eine Vorlage für eine Maturarbeit in der Informatik am Gymnasium Muttenz. Sie dient dazu, die Arbeit schnell und einfach zu starten und sollte einen guten Überblick über die Arbeit bieten.
}

\tableofcontents

\chapter{Einleitung}

KI ist ein Teilgebiet der Informatik was sehr breitgefächert ist und sich mit maschinellem Lernen beschäftigt.
KI kommt in vielen Bereichen in unserem Alltag vor, als Siri in unserem Handy als Naviagtion in unserem Auto und noch vielen weiteren Orten. 
Die KI kann hilfreich sein, uns Fragen beantowrten, uns den Weg zeigen jedoch raubt sie uns eine gewisse Freiheit und da stellt sich meine Frage, ob KI eine Gefahr darstellen kann?

„Erst gestalten wir unsere Werkzeuge, dann gestalten sie uns“ – John Culkin

\section {Was ist KI genau?}
KI ist eine Technologie, die erstmalig wirklich versteht, was sie sagt, sieht und tut und damit ein altes Problem löst.
DIe Künstliche Intelligenz ist die Imitation von menschnlichen Fähigkeiten wie Denken, Lernen...Sie ermöglicht technische Systeme wahrzunehmen und Probleme zu lösen um ein Ziel zu erreichen.

\section {Wo wird KI genutzt?}

\begin{itemize}
    \item Online Shopping und Werbung: Empfehlungen zu geben aufgrund der früheren Käufe oder Produktionssuchen.
    Das führt dazu das User eher was einkaufen und länger auf der Webseite gehalten werden können. Das Online-Shoppen nimmt mehr Anspruch in der Gesellschaft und hat nicht nur die positive Aspekte, zu Hause zu bleiben und alles viel schneller zu erledigen.
    Es führt zu Schwierikeiten beim Datenschutz, Reduktion von Arbeitsplätzen, Fehelranfälligkeiten... 
    \item Virtuellen Assistenten: Chatbotws beantowrten Fragen und helfen den Usern eine Lösung zu finden. Man muss nicht lange auf der Warteschleife sein oder auf eine E-Mail der Mitarbeiter warten.
    Die User fühlen sich wohl jedoch werden ihre Meinungen und Gefühle dadurch eingeschränkt.
    \item Fahrzeuge: Verarbeiten Daten parallel oder in der Echtzeit. Die KI-Systeme lassen sich auch nicht ablenken oder ermüden lassen. Nutzen KI um das Ziel zu erreichen, für KI- Sicherheitsfunktionen, Strassenverlauf, Ampeln und vieles mehr.
    Was schwierig ist, ist die Fehler zu finden, die blinden Flecken. 
   
\end{itemize}



\input{chap_methode.tex}

\section{Wie wird KI trainiert und getestet?}

Etwas mit Änderung hier am Ende.

Wenn ich eine Quelle zitieren möchte, kann ich das ganze einfach am Ende des Satzes machen \citep{example}. Oder wie \citet{example} sagt, auch mitten im Text.

\printbibliography

\end{document}
