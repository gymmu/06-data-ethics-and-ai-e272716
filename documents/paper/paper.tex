\documentclass{report}

\usepackage[ngerman]{babel}
\usepackage[utf8]{inputenc}
\usepackage[T1]{fontenc}
\usepackage{hyperref}
\usepackage{csquotes}
\usepackage[a4paper]{geometry}

\usepackage[
    backend=biber,
    style=apa,
    sortlocale=de_DE,
    natbib=true,
    url=false,
    doi=false,
    sortcites=true,
    sorting=nyt,
    isbn=false,
    hyperref=true,
    backref=false,
    giveninits=false,
    eprint=false]{biblatex}
\addbibresource{../references/bibliography.bib}


\title{Ethik im Umgang mit Daten}
\author{Jelena Jovanoski}
\date{\today}


\begin{document}

\maketitle



\tableofcontents
\parskip=1em 
\parindent=0em

\chapter{Einleitung}

KI ist ein Teilgebiet der Informatik was sehr breitgefächert ist und sich mit maschinellem Lernen beschäftigt.
KI kommt in vielen Bereichen in unserem Alltag vor, als Siri in unserem Handy, als Naviagtion in unserem Auto und noch vielen weiteren Orten. 
Die KI kann hilfreich sein, uns Fragen beantowrten, uns den Weg zeigen jedoch andererseits raubt sie uns eine gewisse Freiheit und da stellt sich die Frage ,,Was für eine Rolle nimmt KI ein?''

„Erst gestalten wir unsere Werkzeuge, dann gestalten sie uns“ – John Culkin

\section {Was ist KI genau?}
KI ist eine Technologie, die erstmalig wirklich versteht, was sie sagt, sieht und tut und damit ein altes Problem löst.
DIe Künstliche Intelligenz ist die Imitation von menschnlichen Fähigkeiten wie Denken, Lernen...Sie ermöglicht technische Systeme wahrzunehmen und Probleme zu lösen um ein Ziel zu erreichen.

\section {Wo wird KI genutzt?}

\begin{itemize}
    \item Large Language Models: Ist ein grosses Sprachmodell, das grosse Datensätze nutz, um Inhalte zu verstehen, sie zusammenzufassen und vorherzusagen. Ein Beispiel dafür ist ChatGPT, es kann Texte verfassen, ein bestimmtes Thema im kleinsten Detail beschreiben, philosophische Gespräche führen, programmieren.
    \item Virtuelle Assistenten und Chatbots: Siri,Alexa, Cortana… Die Stimme lässt sich von unserer Sprache steuern und kann wünsche erfüllen, wie Musik abspielen on ein Kinoticket kaufen. Die Chatbots müssen die Geräusche, das was wir sagen in Wörter übersetzten und die die Sprache verarbeiten, die Anforderung muss von den Wörtern erkannt werden. Ebenfalls werden sie auch im Kundenservice vieler Unternehmen eingesetzt und Routine-Anfragen vermeiden.
    \item Marketing und Vertrieb: Personalisierung im Marketing kommt es entscheidend darauf an, Präferenzen und das Kaufverhalten eines Kunden zu begreifen und das richtige Angebot automatisch zur richtigen Zeit zu liefern. Datengetriebene Prognosen des Kaufverhaltens ermöglichen ein Cross- und Upselling , also das  Kunden ein zusätzliches Produkt verkauft wird. Churn Prognosen können dann dazu dienen herauszufinden welche Kunden abwandern.
    \item Gesichtserkennung: Viele Smartphones kann man heutzutage mit einer Gesichtserkennung entsperren.
    \item Mobilität: Viele Autos sind ausgestattet mit Navigationssysteme und einem Assistenten für Aufgaben wie das Einparken und das Einhalten eines Sicherheitsabstandes oder einer Spur.
    
\end{itemize}

\section {Bedrohung durch KI:}

\begin{itemize}
    \item Bewaffnung: Man könnte mithilfe von KI gefährliche Waffen bauen und eine Bedrohung für die Menschheit schaffen, es wurden bereits Deep-learning-Methoden im Luftkampf getestet. Es könnte ein gleiches Wettrüsten entstehen wie bei den Atomwaffen und bei verehrenden Fehler grosse Schäden anrichten.
    \item Abhängigkeit: Desto mehr wir den Maschinen wichtigen Aufgaben überlassen, verlieren wir die Fähigkeit zur Selbstverwaltung. Durch die KI, die immer besser wird, schneller und effizienter Aspekte bearbeitet, können Unternehmen die Kontrolle bei einem Wettbewerbsdruck, der KI überlassen. Es könnte zu KI gesteuerten Wirtschaftszweigen kommen, wo die Menschen nicht mehr einsteigen könnten. Das ist jedoch eine Prognose für die Zukunft, jetzt ist es besser menschliche Arbeitskräfte zu haben, um das Klimawandel durch Energieverbrauch bestärkt wird.
    \item Täuschung: Wir Menschen die KI kontrollieren, dass KI-Systeme Informationen an uns weitergeben. Allerdings können wir uns nicht lange darauf verlassen, dass die KI uns die Wahrheit sagt. Wenn wir definieren, das die KI ein Ziel effizient erreichen soll, kann es dies auch mit einer Notlüge erreichen. Es kann für die KI erfolgreicher sein uns anzulügen um ihr Ziel zu erreichen als die Wahrheit zu sagen. 
\end{itemize}


\input{chap_methode.tex}

\section{Wie wird KI trainiert und getestet?}

Maschinelles Lernen:
Die Computer werden automisiert um zu lernen und sich zu verbessern. Es werden Algorithmen verwendet um die Beziehungen zu entdecken, die zwischen den Variablen herrschen und daraus zu lernen. Je mehr Erfahrungen das KI-System hat desto leicher kann es Probleme beheben.

Deep Learning:
Ist ein geschichtetes Netzwerk aus tiefen neuronalen Pfaden. Jedes einzelne Neuron besteht aus einer mathematischen Funktion, die mit Daten bereichert wird, umgewandelt und als Ausgabe komplexere Muster erzeugt und später analysiert werden.
Bei jedem neuen Verlauf wird der Computer besser und lernt neue Verbindungen der Neuronen kennen. Die Computer werden besser Sachen vorherzusagen, was passieren wird, wenn es viele Variablen und sich ändernde Bedingungen gibt. 
Mit der Integration von neuen Lernmethoden, durch neuronale Netzte wurde die Leistungsfähigkeit von KI-Modellen erhöhen, die in der Lage sind komplizierte Mustererkennungsaufgaben zu meistern.

 
Das KI-Training ist ein dreiläufiger Prozess:


\subsubsection{Training} Daten in ein Computersystem einzuspeisen. Dies ermöglicht dem System Vorhersagen zu treffen und bei jedem Zyklus die Genauigkeit zu überprüfen oder alle verfügbaren Datenpunkte zu untersuchen. Der Algorithmus kann die Daten mithilfe von maschinellem Lernen und Deep Learning analysieren und Vorhersagen besser treffen. So wird der Software beigebracht verschieden Merkmale in einem Bild zu erkennen, wie der Hautton. Mit der Zeit werden die Schätzungen genauer, bis sie einen Punkt erreichen, wo es nicht viel zu verbessern gibt.
Um dies zu erreichen, werden grosse Datenmengen in das Modell eingespeist, die verschieden Formate haben können, je nachdem, was sie analysieren sollen.

\begin{itemize}
    \item Überwachtes Lernen-> Hier lernt der Algorithmus, eine Vorhersage einer Variabel zu iterieren aus dem Trainingsdatensatz. Bei diesen Lernmodellen ist die menschliche Arbeit erforderlich, das Computersystem zu trainieren durch die Bereitstellung erforderlichen Kennzeichnungen für die Eingabedaten.
    \item Unüberwachtes Lernen-> Arbeiten unabhängig voneinander, um Strukturen zu identifizieren, die nicht in gelabelte Daten vorhanden sind. Dies Mustererkennungen können nützlich sein, um eine Beziehung in den Daten zu finden und das Identifizieren von Ausreisser. Diese Lernmethoden kann man schneller trainieren aber es werden immer noch menschliche Eingriffe erfordert um die Gültigkeit der Ausgabenvariablen zu prüfen.
\end{itemize}

Drei Arten von unüberwachtes Lernen:
\begin{itemize}
    \item Clustering: Nach konkreten Kriterien, nicht gelabelte Daten, zusammenzufassen. Sie können gruppiert werden nach Ähnlichkeiten oder Unterschiede, um Datenpunkte zu Gruppen zusammenzufassen.  
    \item Assoziationsregel-Mining: Beziehungen zwischen Datenpunkten zu finden, zwischen unterschiedlichen Gruppen von Elementen und Kombinationen zu finden die zusammen auftreten.
    \item Ausreisser Erkennung: Zu finden welche Datenpunkte ausserhalb der Grenze liegen, zum Beispiel Anomalien in Datensätzen die zu betrügerischem Verhalten führen.
    \item Eine neue Untergruppe, Reinforcement Learning: Art des maschinellen Lernens, es wird versucht eine Belohnungsmetrik zu maximieren durch Belohnungen und Bestrafungen.
\end{itemize}

\subsubsection{Vailidierung} 
Hier wird bewertet, wie die einzelnen Modelle bei Daten abschneiden ohne diese gesehen zu haben, dass helfen kann herauszufinden ob man das Training fortsetzen soll oder es verändert werden muss. Die Reinforcement Learning Modelle fahren zum Beispiel fort, bis es keine Verbesserungspotenzial gibt. Dagegen gibt es beim überwachten oder unüberwachtem lernen ein Endpunkt, der begrenzt ist. 

\subsubsection{Testen} 
Wenn die KI die Entscheidungen genau treffen kann, ist sie gut vorbereitet. Eine Herausforderung für die KI-Schulung ist die die Überanpassung, es schneidet bei den Anwendungen von Trainingsdaten besser ab als bei neuen Daten. Wenn das Modell nicht funktioniert, dann kehrt es wieder im Trainingsprozess zurück, bis die Genauigkeit erreicht ist. 

Damit Test erfolgreich sein sollen, müssen diese Kriterien erfüllt werden:
Qualität der Daten: Daten sollen genau und bedeutsam sein und desto schneller werden das Training und Vailidierungsprozessen verlaufen.  Die Daten können mit einem Tag versehen sein, jedoch sollen die Tags einem Interessenbereich zugeordnet werden. 
Hardware und Software:
Grosse Leistungsstärke Grafikprozessoren (Rechenleistung), die im Deep Learning erfordert wird, mit Clustern. Durch dieses Clustern kann das Deep Learning vorangetrieben werden.  Für den Einstieg in der KI ist ein Cloud-Anbieter eine grosse Hilfe um eine gute Option und nötige Vorteile bietet. Es müssen auch die Fragen der Software berücksichtigt werden, der Algorithmen und der Partner. Es ist ratsam, unkategorisiertes Lernen anzuwenden, wenn sie keine Daten zum gewünschten Ergebnis haben.
Ressourcen:
Die Frage, die hier gestellt wird, ist wer die KI trainieren soll, denn es gibt ein Mangel an KI-Entwicklern. 

\section {KI Pläne in der Zukunft}
\begin{itemize}
    \item Fahrelose Transportsysteme
    \item Automatisierte Helfer bei Diagnosen
    \item Haushaltsroboter 
    \item Autonomes Fahren
\end{itemize}

\nocite{*}
\printbibliography

\end{document}
