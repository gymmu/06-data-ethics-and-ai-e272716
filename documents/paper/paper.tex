\documentclass{report}

\usepackage[ngerman]{babel}
\usepackage[utf8]{inputenc}
\usepackage[T1]{fontenc}
\usepackage{hyperref}
\usepackage{csquotes}
\usepackage[a4paper]{geometry}

\usepackage[
    backend=biber,
    style=apa,
    sortlocale=de_DE,
    natbib=true,
    url=false,
    doi=false,
    sortcites=true,
    sorting=nyt,
    isbn=false,
    hyperref=true,
    backref=false,
    giveninits=false,
    eprint=false]{biblatex}
\addbibresource{../references/bibliography.bib}


\title{Ethik im Umgang mit Daten}
\author{Jelena Jovanoski}
\date{\today}


\begin{document}

\maketitle



\tableofcontents

\chapter{Einleitung}

KI ist ein Teilgebiet der Informatik was sehr breitgefächert ist und sich mit maschinellem Lernen beschäftigt.
KI kommt in vielen Bereichen in unserem Alltag vor, als Siri in unserem Handy, als Naviagtion in unserem Auto und noch vielen weiteren Orten. 
Die KI kann hilfreich sein, uns Fragen beantowrten, uns den Weg zeigen jedoch andererseits raubt sie uns eine gewisse Freiheit und da stellt sich die Frage ,,Was für eine Rolle nimmt KI ein?''

„Erst gestalten wir unsere Werkzeuge, dann gestalten sie uns“ – John Culkin

\section {Was ist KI genau?}
KI ist eine Technologie, die erstmalig wirklich versteht, was sie sagt, sieht und tut und damit ein altes Problem löst.
DIe Künstliche Intelligenz ist die Imitation von menschnlichen Fähigkeiten wie Denken, Lernen...Sie ermöglicht technische Systeme wahrzunehmen und Probleme zu lösen um ein Ziel zu erreichen.

\section {Wo wird KI genutzt?}

\begin{itemize}
    \item Large Language Models: Ist ein grosses Sprachmodell, das grosse Datensätze nutz, um Inhalte zu verstehen, sie zusammenzufassen und vorherzusagen. Ein Beispiel dafür ist ChatGPT, es kann Texte verfassen, ein bestimmtes Thema im kleinsten Detail beschreiben, philosophische Gespräche führen, programmieren.
    \item Virtuelle Assistenten und Chatbots: Siri,Alexa, Cortana… Die Stimme lässt sich von unserer Sprache steuern und kann wünsche erfüllen, wie Musik abspielen on ein Kinoticket kaufen. Die Chatbots müssen die Geräusche, das was wir sagen in Wörter übersetzten und die die Sprache verarbeiten, die Anforderung muss von den Wörtern erkannt werden. Ebenfalls werden sie auch im Kundenservice vieler Unternehmen eingesetzt und Routine-Anfragen vermeiden.
    \item Marketing und Vertrieb: Personalisierung im Marketing kommt es entscheidend darauf an, Präferenzen und das Kaufverhalten eines Kunden zu begreifen und das richtige Angebot automatisch zur richtigen Zeit zu liefern. Datengetriebene Prognosen des Kaufverhaltens ermöglichen ein Cross- und Upselling , also das  Kunden ein zusätzliches Produkt verkauft wird. Churn Prognosen können dann dazu dienen herauszufinden welche Kunden abwandern.
    \item Gesichtserkennung: Viele Smartphones kann man heutzutage mit einer Gesichtserkennung entsperren.
    \item Mobilität: Viele Autos sind ausgestattet mit Navigationssysteme und einem Assistenten für Aufgaben wie das Einparken und das Einhalten eines Sicherheitsabstandes oder einer Spur.
    
\end{itemize}



\input{chap_methode.tex}

\section{Wie wird KI trainiert und getestet?}

Maschinelles Lernen:
Die Computer werden automisiert um zu lernen und sich zu verbessern. Es werden Algorithmen verwendet um die Beziehungen zu entdecken, die zwischen den Variablen herrschen und daraus zu lernen. Je mehr Erfahrungen das KI-System hat desto leicher kann es Probleme beheben.

Deep Learning:
Ist ein geschichtetes Netzwerk aus tiefen neuronalen Pfaden. Jedes einzelne Neuron besteht aus einer mathematischen Funktion, die mit Daten berreichert wird, umgewandelt und als Ausgabe komplexere Muster erzeugt und später  analysiert werden.
Bei jedem neuen Verlauf wird der Computer besser und lernt neue Verbindungen der Neuronen kennen. Die Computer werden besser Sachen vorherzusagen und   
 
Das KI-Training ist ein dreiläufiger Prozess:

 

Wenn ich eine Quelle zitieren möchte, kann ich das ganze einfach am Ende des Satzes machen \citep{example}. Oder wie \citet{example} sagt, auch mitten im Text.

\printbibliography

\end{document}
