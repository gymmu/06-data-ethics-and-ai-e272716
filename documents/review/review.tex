\documentclass{article}

\usepackage[ngerman]{babel}
\usepackage[utf8]{inputenc}
\usepackage[T1]{fontenc}
\usepackage{hyperref}
\usepackage{csquotes}

\usepackage[
    backend=biber,
    style=apa,
    sortlocale=de_DE,
    natbib=true,
    url=false,
    doi=false,
    sortcites=true,
    sorting=nyt,
    isbn=false,
    hyperref=true,
    backref=false,
    giveninits=false,
    eprint=false]{biblatex}
\addbibresource{../references/bibliography.bib}

\title{Review des Papers "Ethik im Umgang mit Daten" von Zoe Donghi \dots}
\author{Jelena Jovanoski}
\date{\today}

\begin{document}
\maketitle

\abstract{

    \noindent 
    Ich schreibe mein Review über Zoe Donghi’s Arbeit,, Ethik im Umgang von Daten. ’’ In ihrer Arbeit geht es um Vorkommen von KI, die Funktion, die Bedeutung, die Risiken und Grenzen und die Methoden. 
    Es wird gut beschrieben, was KI ist und die negativen, wie positiven Einflüsse von KI in unserem Alltag. Ich finde die Arbeit an sich toll, ich habe neue Sachen gelernt, eine andere Beschreibung von KI, jedoch wäre es auch spannend gewesen etwas vom Training oder vom Testen der KI zu erfahren oder mehr darüber, wie KI funktioniert.
}


\printbibliography

\end{document}
